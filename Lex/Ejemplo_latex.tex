\documentclass{article}
\usepackage[utf8]{inputenc}
\usepackage{lmodern}
\usepackage[T1]{fontenc}
\usepackage[spanish,activeacute]{babel}
\usepackage{mathtools}
\usepackage{hyperref}
\usepackage{graphicx}
\usepackage{soul}

\title{Ejemplo para la práctica lex}
\author{Paula Villanueva}
\date{\today}

\begin{document}

\maketitle

\section{Formato del texto}
\subsection{Negrita}
En esta sección podemos observar que podemos poner palabras en \textbf{negrita}.

\subsection{Cursiva}
En esta sección podemos observar que podemos poner palabras en \textit{cursiva}.

\subsection{Subrayado}
En esta sección podemos observar que podemos poner palabras en \underline{subrayadas}.

\subsection{Tachado}
En esta sección podemos observar que podemos poner palabras \textst{tachadas}.

\section{Comentarios}
La siguiente oración no se podrá ver porque es un comentario, pero en el código si se podrá contemplar.

% Hola, esto es un comentario

\section{Lista}
Ahora vamos a ver una lista de items:

\begin{itemize}
\item Este es el primer item
\item Este es el segundo item
\item Este es el tercer item y último
\end{itemize}

\section{Fórmula}
Vamos a ver la siguiente fórmula:

\begin{equation}
\sum_{i=0}^{n} x_i + i = \phi
\end{equation}



\section{Ecuación}

También podemos escribir una ecuación, como la siguiente $\frac{2}{4} = \frac{1}{2}$, en la misma linea. O también, podemos escribirla
en la siguiente linea:
$$\frac{2}{4} = \frac{1}{2}$$

\section{Links}

Podemos ver el siguiente link con descripción: \href{http://www.sharelatex.com}{Something Linky} o sin descripción url: \url{http://www.sharelatex.com}

\section{Imágenes}
En esta sección podemos observar una bonita imagen del universo.

\includegraphics{universo.jpg}

\end{document}
